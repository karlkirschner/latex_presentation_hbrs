\documentclass[11pt,vertical]{beamer}
\usepackage[utf8]{inputenc}
\usepackage[english]{babel}
\usepackage{balance}
\usepackage{microtype}
\usepackage{booktabs, threeparttable}
\usepackage{color}
\usepackage{multimedia}

\usetheme{_hbrs_v1} %% read beamertheme_hbrs_v1.sty

\newcommand{\mytitlefoot}{\footnotesize Presentation Title}
\title{ \textcolor{HBRS}{\Huge\textbf{Title of Your\\ \vspace{1cm} Presentation}} }

\author{YOUR NAME}
\date{Date}

\begin{document}

\frame{\titlepage}

\begin{frame}{\small \textbf{SLIDE 1}}
	
	\begin{itemize}
	    \item \Huge  $\backslash$Huge
	    \item \huge  $\backslash$huge
	    \item \LARGE $\backslash$LARGE
	    \item \Large $\backslash$Large
	    \item \large $\backslash$large
	    \item \normalsize $\backslash$normalsize
	    \item \small $\backslash$small
	    \item \footnotesize $\backslash$footnotesize
	    \item \scriptsize $\backslash$scriptsize
	    \item \tiny $\backslash$tiny
  \end{itemize}

\end{frame}


\begin{frame}{\small \textbf{SLIDE 2}}
	\centering

	Customized horizontal lines (from beamertheme\_hbrs\_v1.sty):
	\lhline
	\mhline
	\shline

	\vfill

	The following is for vertical alignment of two images:
	\lhline
	\includegraphics[valign=m, width=3.0cm]{logo_hbrs}\hspace{1cm}
	\includegraphics[valign=m, width=5.0cm]{logo_hbrs}
	%% Altnerative approach:
%	$\vcenter{\hbox{\includegraphics[height=0.7cm]{logo_hbrs}}}$\hspace{1cm}
%	$\vcenter{\hbox{\includegraphics[height=1.2cm]{logo_hbrs}}}$
	\lhline

	\vfill
	\includegraphics[width=0.5\textwidth]{logo_hbrs}
\end{frame}

\begin{frame}[fragile]{\small \textbf{Example - Docker (one possible approach)}}
\small

\vfill
Example of a \textbf{\textcolor{HBRS}{environment.yml}} adding code to your presentation.
%\tiny
\begin{lstlisting}
name: compchem_students
channels:
  - defaults
dependencies:
  - ipykernel
  - <@\textcolor{red}{matplotlib}@>
  - <@\textcolor{red}{openmm}@>
#  - <@\textcolor{red}{ambertools}@>
#  - pip
prefix: /mnt/sdc1/miniconda3/envs/compchem_students
\end{lstlisting}
\end{frame}

\begin{frame}{\small \textbf{SLIDE 3}}
    \transfade
	\small
	\begin{columns}
	
		\begin{column}[t]{0.4\textwidth}
			\begin{itemize}
				\item Test 1
				\item Test 2
				\item Test 3
				\item Test 4
			\end{itemize}
		\end{column}
		
		\begin{column}[t]{0.4\textwidth}
			\begin{enumerate}
				\item \textcolor{red}{Example 1}
				\item \textcolor{HBRS}{Example 2}
				\item \textcolor{orange}{Example 3}
				\item Example 4
			\end{enumerate}
		\end{column}
		
	\end{columns}
	\vfill
	\lhline
	\LARGE
	TEXT
	\vfill

	\credit{Source:\\
	a) something}

\end{frame}

\begin{frame}{\small \textbf{Example - Student Install of a VM Image}}
	\center
	The following may not appear in TexMaker, but it should show up when the presentation PDF is opened using a PDF viewer (e.g. okular):
	\vfill
    \movie[width=8.5cm, height=5.0cm, poster, showcontrols=true]{}{movie_example.mp4}
\end{frame}

\begin{frame}{\small \textbf{SLIDE 4}}
    \transfade
    A slide that includes a transion to it.

    And includes an MP4 movie (requires \textcolor{HBRS}{multimedia} package)

    \bigskip
    \centering
    \movie[width=3.5cm, height=3.3cm, poster, showcontrols=true]{}{./movie.mp4}

\end{frame}


%% slide transitions (added after \begin{frame} line - see example above)
%% Infor and source: https://tug.ctan.org/macros/latex/contrib/beamer/doc/beameruserguide.pdf

% \transblindshorizontal : Show the slide as if horizontal blinds were pulled away.
% \transblindsvertical : Show the slide as if vertical blinds were pulled away.
% \transboxin : Show the slide by moving to the center from all four sides.
% \transboxout : Show the slide by showing more and more of a rectangular area that is centered on the slide center
% \transcover : Show the slide by covering the content that was shown before
% \transdissolve : Slowly dissolve what was shown before
% \transfade : Show the slide by slowly fading what was shown before.
% \transfly : Show the slide by letting the new content fly in before removing the old slide.
% \transglitter : Glitter sweeps in specified direction
% \transpush : Show the slide by pushing what was shown before off the screen using the new content.
% \transreplace : Replace the previous slide directly (default behaviour).
% \transsplitverticalin : Show the slide by sweeping two vertical lines from the sides inward.
% \transsplitverticalout : Show the slide by sweeping two vertical lines from the center outward.
% \transsplithorizontalin : Show the slide by sweeping two horizontal lines from the sides inward.
% \transsplithorizontalout : Show the slide by sweeping two horizontal lines from the center outward.
% \transwipe : Sweeps single line in specified direction
% \transduration{2} : Show slide specified number of seconds

\end{document}